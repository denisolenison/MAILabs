\documentclass[12pt]{article}

\usepackage{fullpage}
\usepackage{multicol,multirow}
\usepackage{tabularx}
\usepackage{ulem}
\usepackage[utf8]{inputenc}
\usepackage[russian]{babel}
\usepackage{listings}
\usepackage{hyperref}


\begin{document}

\section*{Лабораторная работа №\,1 по курсу криптографии}

Выполнил студент группы М8О-308Б-17 \textit{Иларионов Денис}.

\subsection*{Условие}
Разложить каждое из чисел n1 и n2 на нетривиальные сомножители.\\
Вариант 7.\\
n1=268887320029090028117214498253204095765884136483366193842361283776500643966781,\\
n2=141774678697875076503878344320169483769305800714713500792858319214425694670422365\\
9049475898042715778235153026085212635256089348105695559658585619676085161346482180413\\
6259107185547729368883111388512812700339059708262004996928275687558408584407339919174\\
5402825532617474496569647039364471309183150878711637228946726608456444330507998028604\\
9350362289761393863307795187974797187985957533461476088825816395922558727920330066823\\
211210594296302676261707432217348305112187,

\subsection*{Метод решения}

Я сначала долго думал над тем, как решить данную задачу, искал различные алгоритмы, но они мне показались довольно сложными. Наиболее эффективными алгоритмами для факторизации больших чисел являются метод квадратичного решета и факторизация с помощью эллиптических кривых. Какое-то время я думал, изучал эти алгоритмы. А позже, узнал от одногруппников, что на самом деле есть кое-какая зависимость у сомножителей второго числа. А именно, что первый его сомножитель находится как НОД с числом из соседнего варианта, а чтобы получить второй - нужно поделить число на этот сомножитель. Также, я узнал, что можно использовать любые средства, и это облегчило мне задачу. Есть много известных и более простых методов факторизации, но они работают за $ O( n * \lg{n} ) $ и за $ O( \sqrt{n} ) $. Их вполне хватает для нахождения сомножителей чисел с 6-9 нулями, но тут у нас числа в разы больше. И такие алгоритмы будут выполнять свою работу так долго, что к тому времени Солнце уже успеет превратиться в красного гиганта и поглотить Землю. Поэтому, такие алгоритмы не эффективны для данной ЛР.

Первое число я решил разложить на этом сайте \url{https://www.alpertron.com.ar/ECM.HTM}\\
Второе число нашел уже с помощью своей длинной арифметики из 6 лабораторной по дискретному анализу. Но нужен именно самый эффективный алгоритм деления, ибо даже с делением на подобии бинарного поиска, у меня очень долго искала даже НОД с 1 вариантом, а с деление q\^\ нашло все НОДы за минуту.\\

Я думаю, что целью данной ЛР было не только узнать побольше об алгоритмах факторизации, но и именно разгадать шифр второго числа, найти взаимосвязь. Именно похожие решения и используются в криптографии, а шифр RSA как раз и основан на факторизации.


\subsection*{Результат работы программы}
\begin{verbatim}
268887 320029 090028 117214 498253 204095 765884 136483 366193 842361 283776
500643 966781 (78 digits) = 414 150068 879409 136107 176764 405542 089303 (39
digits) × 649 250936 397607 504492 837402 141095 065227 (39 digits)


 Variant 0 GCD = 1


 Variant 1 GCD = 1


 Variant 2 GCD = 1


 Variant 3 GCD = 1


 Variant 4 GCD = 1


 Variant 5 GCD = 1


 Variant 6 GCD = 1


 Variant 8 GCD = 13047386803258360982718544898036475818102572197915858405851096
 848067467531800758251299143837940990002563420660299933504725395859764575192257
 231951973654902109838773403174196838688508746956481136954975656279172115606067
 161186055342248910459683542763941795236487891843320360172073464935554329358004
 8106131166649


 Variant 9 GCD = 1


 Variant 10 GCD = 1


 Variant 11 GCD = 1


 Variant 12 GCD = 1


 Variant 13 GCD = 1


 Variant 14 GCD = 1


 Variant 15 GCD = 1


 Variant 16 GCD = 1


 Variant 17 GCD = 1


 Variant 18 GCD = 1


 Variant 19 GCD = 1


First Divider = 130473868032583609827185448980364758181025721979158584058510968
4806746753180075825129914383794099000256342066029993350472539585976457519225723
1951973654902109838773403174196838688508746956481136954975656279172115606067161
1860553422489104596835427639417952364878918433203601720734649355543293580048106
131166649


Second Divider = 10866135942445523506369738829513009804011846137158787994185181
3508603646007476833795989690941616410019713882182770139000415549937936708389778
01293355191763

Vse Okey ;)

\end{verbatim}

\newpage

\subsection*{Листинг программного кода (только файл main.cpp)}
\begin{lstlisting}

#include "BigInt.h"
#include <iomanip>
using namespace std;

NBigInt::TBigInt GCD(NBigInt::TBigInt a, NBigInt::TBigInt b) {
	while (a != 0) {
		if (a < b) {
			NBigInt::TBigInt tmp = a;
			a = b;
			b = tmp;
		}
		NBigInt::TBigInt k = a / b;
		a = a - b * k;
	}
	return b;
}

int stoi(const string& str) {
	int tmp = 0;
	for (int i = 0; i < str.size(); ++i) {
		tmp = tmp * 10 + (str[i] - '0');
	}
	return tmp;
}

int main() {

	freopen("input.txt", "r", stdin);
	string stringNum1;
	cin >> stringNum1;
	NBigInt::TBigInt ishodnick(stringNum1);


	for (int i = 0; i < 19; ++i) {
		string stringNum2;
		cin >> stringNum2;
		NBigInt::TBigInt number2(stringNum2);
		NBigInt::TBigInt res = GCD(ishodnick, number2);
		if (i >= 7) {
			cout << "\n\n Variant " << i + 1;
		}
		else {
			cout << "\n\n Variant " << i;
		}
		cout << " GCD = " << res << "\n\n";
	}


	
	NBigInt::TBigInt numberOne("1304738680325836098271854489803647
	58181025721979158584058510968480674675318007582512991438379409
	90002563420660299933504725395859764575192257231951973654902109
	83877340317419683868850874695648113695497565627917211560606716
	11860553422489104596835427639417952364878918433203601720734649
	355543293580048106131166649");
	NBigInt::TBigInt numberTwo = ishodnick / numberOne;
	cout << "\n\nFirst Divider = " << numberOne << "\n";
	cout << "\n\nSecond Divider = " << numberTwo << "\n\n";
	
	if ((numberOne * numberTwo) == ishodnick) {
		cout << "Vse Okey ;)\n";
	}
	else {
		cout << "Smthng wrong\n";
	}
	cout << "\n\n";

	while (true) { 
		int wait;
		cin >> wait;
	}
	return 0;

}


\end{lstlisting}

\newpage

\subsection*{Выводы}
Факторизация целых чисел - это довольно важный аспект математики. Любое число можно разложить на простые множители, причем, единственным образом. Есть очень много алгоритмов факторизации, но для очень больших чисел еще эффективный алгоритм не придумали. Даже алгоритмы факторизации с помощью эллиптических кривых или квадратичного решета будут работать вечность при факторизации чисел с тысячью/десятью тысячами цифр.\\
Факторизация больших чисел является задачей большой сложности. Не существует никакого известного способа, чтобы решить эту задачу быстро. Также, после выполнения данной лабораторной работы, я понял, что алгоритм шифрования RSA (который как раз и основан на факторизации) является довольно эффективным и безопасным, и на него можно полагаться.

\end{document}