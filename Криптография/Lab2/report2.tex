\documentclass[12pt]{article}

\usepackage{fullpage}
\usepackage{multicol,multirow}
\usepackage{tabularx}
\usepackage{ulem}
\usepackage[utf8]{inputenc}
\usepackage[russian]{babel}
\usepackage{graphicx}
\DeclareGraphicsExtensions{.jpg}

\renewcommand{\labelenumii}{\arabic{enumi}.\arabic{enumii}.}


\begin{document}

\section*{Лабораторная работа №\,2 по курсу криптографии}

Выполнил студент группы М8О-308Б-17 \textit{Иларионов Денис}.

\subsection*{Условие}

\begin{enumerate} 
\item Сгенерировать OpenPGP-ключ и самоподписанный сертификат (например, с помощью дополнения Enigmail к почтовому клиенту thunderbird).
\item Установить связь с преподавателем и с хотя бы с одним одногруппником, используя созданный ключ, следующими действиями:
\begin{enumerate} 
\item Прислать от своего имени по электронной почте сообщение, во вложении которого поместить свой открытый ключ.
\item Дождаться письма, в котором отправитель вам пришлёт свой сертификат открытого ключа.
\item Выслать сообщение, зашифрованное на ключе отправителя.
\item Расшифровать письмо своим закрытым ключом.
\item Убедиться, что ключу абонента можно доверять путём сравнения отпечатка ключа или ключа целиком, по доверенным каналам связи.
\end{enumerate} 
\item Собрать подписи под своим ключом.
\begin{enumerate}
\item Подписать сертификат открытого ключа одногруппника и преподавателя своим ключом.
\item Выслать почтой сертификат полученный в п.3.1 его владельцу.
\item Собрать 10 подписей одногруппников под своим сертификатом.
\item Прислать преподавателю (желательно почтой) свой сертификат, с 10-ю или более подписями одногруппников.
\end{enumerate}
\end{enumerate}


\subsection*{Метод решения}
Файлы:\\
0x75F8B2FF246DDB41.asc -- мой ключ.\\
0xA67701829D9C5DE4.asc -- подписанный ключ преподавателя.\\
\newpage

Я сгенерировал свой ключ и отправил его преподавателю:\\
\includegraphics[width=\linewidth]{Screen1.jpg}\\

А также отправил ему зашифрованное сообщение на своем ключе:\\
\includegraphics[width=\linewidth]{Screen2.jpg}\\

Также я отправил свой ключ одногруппнику, чтобы он подписал мне его:\\
\includegraphics[width=\linewidth]{Screen3.jpg}\\

И зашифровал сообщение на своем ключе:\\
\includegraphics[width=\linewidth]{Screen4.jpg}\\
\newpage

Вскоре, мне пришел ответ:\\
\includegraphics[width=\linewidth]{Screen5.jpg}\\

Я отправил одногруппнику сообщение, зашифрованное его закрытым ключом:\\
\includegraphics[width=\linewidth]{Screen6.jpg}\\

Собственно, вот и шифр, также, я подписал его ключ, а он мой:\\
\includegraphics[width=\linewidth]{Screen7.jpg}\\
\includegraphics[width=\linewidth]{Screen8.jpg}\\
\newpage

И в конце концов, я собрал 12 подписей от своих однокурсников:\\
\includegraphics[width=\linewidth]{Screen9.jpg}\\

А также я отправил преподавателю сообщение, зашифрованное на его ключе:\\
\includegraphics[width=\linewidth]{Screen10.jpg}\\
\includegraphics[width=\linewidth]{Screen11.jpg}\\

\subsection*{Выводы}
В ходе данной лабораторной работы мне удалось научиться пользоваться шифрованием и подписью на примере pgp и почты на основе клиента thunderbird. Основные трудности при выполнении работы были связаны с тем, что я пытался разобраться в интерфейсе программы, но позже, я понял, что нужно делать. Также, пришлось какое-то время подождать, чтобы мне ответили одногруппники и подписали мой ключ. У нас была специальная беседа, где мы обменивались почтами и писали письма на них с просьбами подписать ключ. Взамен, я подписывал ключи людей, которые подписывали мой. Так что, с этим особых проблем не возникло. В итоге, мне удалось 12 человек убедить подписать мой сертификат. А так, ничего особо сложного в работе не было, нужно лишь было отправлять много сообщений и разбираться с интерфейсом. \\

Механизм работы pgp показался мне достаточно интересным. И сообщения прочитать смогут только те, кто имеют нужный ключ. Таким образом, можно обсуждать что-то очень секретное по почте, и никто не сможет узнать, что там написано.

\end{document}